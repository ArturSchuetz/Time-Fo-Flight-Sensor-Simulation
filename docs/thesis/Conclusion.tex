\documentclass[thesis.tex]{subfiles}
\begin{document}

\chapter{Fazit}
\label{chap:conclusion}

\section{Zusammenfassung}

Im Rahmen dieser Arbeit wurden drei Time-of-Flight Kamerasysteme im Detail untersucht und anschließend simuliert. Während der Analyse der Kameras wurden mehrere Fehler untersucht, die durch interne und externe Einflüsse verursacht wurden. Dabei ist die Bedeutung des systematischen Fehlers deutlich geworden, dem alle untersuchten Kamerasysteme unterworfen waren. Ebenfalls wurde der Einfluss des Lens Scatterings untersucht, der sich auf das gesamte Tiefenbild erstreckt. 

Die Simulation der Time-of-Flight Sensoren wurde auf der Grafikkarte mittels OptiX implementiert, während zur Simulation der Oberflächen das Mikrofacetten Modell verwendet wurde. Die Simulation verwendete das von Torrance und Sparrow \cite{bib:TorranceSparrow1967} entwickelte Modell für raue, glänzende Oberflächen und das von Oren und Nayar \cite{bib:OrenNayar1994} entwickelte Modell für raue, diffuse Oberflächen. Zur Simulation der Lichtausbreitung wurde ein Path Tracing Algorithmus implementiert. In der Simulation wurde die Linsenverzeichnung berücksichtigt, indem zunächst die Infrarotkameras der Time-of-Flight Systeme mittels eines Schachbrettmusters kalibriert und zur Erzeugung der Strahlen verwendet wurden. Der systematische Fehler wurde berücksichtigt, indem die Lichtausbreitung unter Berücksichtigung der Amplitudenmodulation der Infrarot LEDs simuliert wurde. Der Lens Scattering Fehler wurde dabei eingemessen und anschließend auf das Bild angewandt.

Abschließend wurde in einer Evaluation gezeigt, dass die Simulation der Time-of-Flight Sensoren den analysierten Fehlern unterworfen war. Dabei wurde die Simulation im Wesentlichen den Tiefenbildern der Kinect v2 gegenübergestellt und der Einfluss der indirekten Beleuchtung, der LED Positionierung und des Lens Scattering untersucht.  

\section{Kritische Reflextion}

In Kombination mit der Simulation des Lens Scattering und der Berücksichtung der LED Position konnten Effekte in Tiefenbildern erzeugt werden, die im vorangegangenen Kapitel beobachtet wurden. Der systematische Fehler wurde erstmals physikalisch plausibel simuliert, während andere Arbeiten diesen Fehler erzeugten, indem sie diese Abweichungen einmessen und anschließend auf das Endresultat der Simulation aufrechnen. Für die Simulation des Lens Scattering und der Linsenverzeichnung wurde allerdings ebenfalls auf Messungen existierender Kameras zurückgegriffen.

Es wurde zwar gezeigt, dass die Simulation einen systematischen Fehler verursacht, der allerdings von den analysierten Kameras abweicht, was darauf zurückzuführen ist, dass die interne Funktionsweise und Implementierung der Schätzung der Phasenverschiebung nicht bekannt ist und deshalb nur Vermutungen über genutzte Technologien aufgestellt werden konnten. Ausschließlich über die Funktionsweise der Kinect v2 war mehr bekannt, da sich Arbeiten mit der detaillierten Analyse des Systems beschäftigten \cite{bib:Giancola2018}. Die Simulation des Fehlers der Kinect v2 war im Rahmen dieser Arbeit allerdings nicht möglich.

Die Simulation des Multiple Path Fehlers ähnelt den Ergebnissen der vorangegangenen Analyse der Kameras, mit Ausnahme der Abweichung an dem Berührungspunkt der beiden orthogonal zueinander positionierten Oberflächen. Da im Rahmen der Arbeit kein direkter Vergleich eines exakt nachgebildeten Versuchsaufbaus innerhalb der Simulation möglich war, bleibt die Frage offen wie genau die Abweichung der simulierten Tiefenwerte den echten Messungen entspricht. Darüber hinaus waren die Rauheit und die optische Dichte der untersuchten Materialien nicht bekannt und mussten in der Evaluation geschätzt werden, weshalb kein genauer Vergleich möglich war. Neben den Eigenschaften der Oberflächen war außerdem die genutzte Wellenlänge der Infrarot LED nicht bekannt, weshalb keine genauen Reflexionseigenschaften der Strahlung an der Oberfläche bestimmt werden konnten, da diese von der Wellenlänge abhängig sind.

Wie auch in den vorangegangenen Arbeiten wurde auch in dieser Arbeit der Mixed Pixels Fehler simuliert. Im Kontrast zu vergleichbaren Arbeiten wurde hierfür allerdings die Simulation des Lens Scattering Fehlers priorisiert, der den Mixed Pixels Fehler nach sich zieht, was in der Analyse und der Evaluation gezeigt wurde. Vorangegangene Arbeiten verzichteten auf die Simulation des Lens Scattering und näherten ausschließlich den Mixed Pixels Fehler an, indem mehrere Samples pro Pixel berechnet wurden, was laut den Analysen dieser Arbeit den gemessenen Ergebnissen widerspricht.

Durch die Nutzung von OptiX zur Berechnung des Path Tracing Algorithmus war es möglich den Algorithmus zu beschleunigen und Eingaben des Nutzers während der Berechnung zu erlauben, was für den Einsatz einen Vorteil bietet, da mögliche Positionen für das Kamerasystem getestet und das Ergebnis innerhalb kurzer Zeit evaluiert werden kann. Außerdem können Parameter der Kamera zur Laufzeit modifiziert werden, was einer der Gründe in Kellers Dissertation ist, weshalb die Simulation der Lichtausbreitung nicht berücksichtigt wurde. Dieser Vorteil kommt allerdings mit dem Preis, dass die Simulation an Kanten ungenauer wird und der Einfluss der indirekten Beleuchtung daher fehlerhaft berechnet wird.

\section{Ausblick}

Die Simulation eines Time-of-Flight Sensors mittels Path Tracing lieferte Ergebnisse, die mit den Untersuchungen eine gute Übereinstimmung zeigen. Da es sich bei der Simulation der Lichtausbreitung um ein rechenintensives Problem handelt, profitiert der Algorithmus von der Implementierung und der parallelen Ausführung auf der GPU. Allerdings ist die Implementierung, die im Rahmen dieser Arbeit angefertigt wurde nicht echtzeitfähig. Durch das Verzichten auf hohe Pfadlängen kann die Berechnungszeit deutlich reduziert werden, was allerdings eine inkorrekte Berechnung der indirekten Beleuchtung zur Folge hätte. Besonders die hohe Komplexität der verwendeten Beleuchtungsmodelle führt zu langen Ausführungszeiten, die durch die Nutzung von vereinfachten Annäherungen reduziert werden können. In zukünftigen Arbeiten können daher unterschiedliche Modelle zur Beleuchtung der Oberflächen untersucht werden, um die Berechnung zu beschleunigen und eine interaktive Anwendung bei gleichzeitig korrekter Berechnung der Lichtausbreitung zu ermöglichen. Darüber hinaus besteht die Möglichkeit eingemessene BRDFs zu verwenden und die Ergebnisse der genutzten Modelle zu evaluieren.

Im Rahmen dieser Arbeit wurde der Einfluss der Temperatur auf die Tiefenwerte zwar untersucht, allerdings nicht simuliert. Zukünftige Arbeiten könnten sich daher im Detail mit dem Einfluss der Temperatur beschäftigen und diese in der Simulation berücksichtigen. Zusätzlich wurde Bewegungsunschärfe im Rahmen der Arbeit nicht simuliert. OptiX bietet die Möglichkeit Bewegungen während der Berechnung des Bildes zu berücksichtigen, wodurch die Auswirkungen von Bewegungen der Kamera oder Objekte in der Szene in die Simulation einbezogen werden können. Dazu sind allerdings Details über die Architektur der verwendeten Sensorchips notwendig, da diese sich auf die Artefakte auswirken \cite{bib:Lambers2015}.

Während der Untersuchung der Xtion 2 wurde festgestellt, dass die Tiefenwerte durch die künstliche Beleuchtung im Raum beeinflusst werden. Dabei ist zu beobachten, dass die Tiefenwerte zwischen den einzelnen Tiefenbildern stark variieren, was durch das Flimmern der Lampen mit einer Frequenz von 50 Hz verursacht wird. Dieses Flimmern war im Falle der Xtion 2 im Tiefenbild in Form von verfälschten Tiefenwerten erkennbar, während die Kinect v2 und die O3D303 davon unbeeinflusst blieben. Für den Einsatz wäre daher eine Simulation von künstlicher Beleuchtung von Vorteil, damit evaluiert werden kann, wie stark sich die Einflüsse der künstlichen Beleuchtung auf die Tiefenwerte unter verschiedenen Positionierungen auswirken.

\subfilebib % Makes bibliography available when compiling as subfile
\end{document}
